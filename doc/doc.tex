% Begin header
\documentclass[UTF8]{ctexart}
\usepackage{amsmath}
\usepackage{graphicx}
\usepackage{minted}
\usepackage{hyperref}
\usepackage{fontspec}
\usepackage{float}
\usepackage[normalem]{ulem}

\useunder{\uline}{\ul}{}
\hypersetup{hidelinks}
\graphicspath{ {images/} }
\setmonofont{Fantasque Sans Mono}
\setCJKmonofont{Microsoft YaHei}
\setminted[cpp]{linenos, breaklines, tabsize=2}
\newmintinline{cpp}{}
\newmintinline[code]{text}{}
\newcommand{\cppsubp}[1]{\subparagraph{\cppinline/{#1}/}\mbox{}\\}
\newcommand{\myfigure}[2]{\begin{figure}[H]\caption{{#1}}\includegraphics[width=\textwidth,height=80mm,keepaspectratio]{{#2}}\centering\end{figure}}
% End header

\title{计算机图形学实验报告}
\date{\today}
\author{沈俞霖}

\begin{document}
  \maketitle
  \vspace{80mm}
  \begin{flushright}

  \textbf{成员}     \makebox[7em][l]{软件51 沈俞霖}

  \textbf{学号}     \makebox[7em][l]{2151601013}

  \textbf{提交日期} \makebox[7em][l]{\today}

  \textbf{联系电话} \makebox[7em][l]{13679119978}

  \end{flushright}
  \newpage

  \tableofcontents
  \newpage

  \section{几何图形的绘制}
    \subsection{实验目的}
      利用OpenGL绘制简单的几何图形
    \subsection{实验内容}
      绘制点、直线(实线和虚线)、多边形(区分正反面)、多边形镂空
    \subsection{实验代码}
      \inputminted{cpp}{../src/shape.cpp}
    \subsection{实验结果}
      \myfigure{几何图形的绘制}{../img/shape.png}
    \subsection{小结}
      通过这次实验,我学会了初始化OPENGL和使用OPENGL绘制简单的2D图形,初步认识了OPENGL1.0的状态机模型

  \section{着色方式}
    \subsection{实验目的}
      掌握OpenGL两种颜色模型:RGBA模型和颜色索引模型
    \subsection{实验内容}
      \begin{itemize}
        \item 利用RGBA模型对多边形进行着色
        \item 利用颜色索引模型对多边形进行着色
        \item 分别设置平滑和单色两种方式绘制多边形,观察两种方式的区别。
      \end{itemize}
    \subsection{实验代码}
      \inputminted{cpp}{../src/color.cpp}
    \subsection{实验结果}
      \myfigure{着色方式}{../img/color.png}
    \subsection{小结}
      通过这次实验,我巩固了计算机内颜色表示的知识,初步认识了OPENGL的颜色模型的概念和使用方法

  \section{三维场景的绘制}
    \subsection{实验目的}
      掌握OpenGL的模型变换和视图变换、投影变换与视口变换,熟悉矩阵堆栈的操作。
    \subsection{实验内容}
      绘制太阳、地球与月亮,采用OpenGL各种变换,绘制它们之间的正确三维关系。
    \subsection{实验代码}
      \inputminted{cpp}{../src/threed.cpp}
    \subsection{实验结果}
      \myfigure{三维场景的绘制}{../img/threed.png}
    \subsection{小结}
      通过这次实验,我学会了计算机图形学常用的3D视图变换的具体实现,学会了使用OPENGL编写3D程序

  \section{纹理映射}
    \subsection{小组成员}
      \begin{itemize}
        \item 软件51,沈俞霖,学号2151601013
        \item 软件51,李南辰,学号2151601010
      \end{itemize}
    \subsection{实验目的}
      掌握OpenGL的纹理映射功能,绘制真实感图形。
    \subsection{实验内容}
      采用绘制圆球和纹理映射的方式,实现地球仪的绘制,并加入光照、纹理等信息,使得所绘制的地球仪显得比较逼真。
    \subsection{实验代码}
      \inputminted{cpp}{../src/texture.cpp}
    \subsection{实验结果}
      \myfigure{纹理映射}{../img/texture.png}
    \subsection{小结}
      通过这次实验我锻炼了我的综合能力,学会了使用OPENGL编写中等规模程序的方法和技巧
\end{document}
